%% abtex2-modelo-projeto-pesquisa.tex, v-1.9.6 laurocesar
%% Copyright 2012-2016 by abnTeX2 group at http://www.abntex.net.br/ 
%%
%% This work may be distributed and/or modified under the
%% conditions of the LaTeX Project Public License, either version 1.3
%% of this license or (at your option) any later version.
%% The latest version of this license is in
%%   http://www.latex-project.org/lppl.txt
%% and version 1.3 or later is part of all distributions of LaTeX
%% version 2005/12/01 or later.
%%
%% This work has the LPPL maintenance status `maintained'.
%% 
%% The Current Maintainer of this work is the abnTeX2 team, led
%% by Lauro César Araujo. Further information are available on 
%% http://www.abntex.net.br/
%%
%% This work consists of the files abntex2-modelo-projeto-pesquisa.tex
%% and abntex2-modelo-references.bib
%%

% ------------------------------------------------------------------------
% ------------------------------------------------------------------------
% abnTeX2: Modelo de Projeto de pesquisa em conformidade com 
% ABNT NBR 15287:2011 Informação e documentação - Projeto de pesquisa -
% Apresentação 
% ------------------------------------------------------------------------ 
% ------------------------------------------------------------------------

\documentclass[
% -- opções da classe memoir --
40pt,				% tamanho da fonte
openright,			% capítulos começam em pág ímpar (insere página vazia caso preciso)
oneside,			% para impressão em recto e verso. Oposto a oneside
a4paper,			% tamanho do papel. 
% -- opções da classe abntex2 --
chapter=TITLE,		% títulos de capítulos convertidos em letras maiúsculas
sumario=tradicional,
english,			% idioma adicional para hifenização
]{abntex2}

% ---
% PACOTES
% ---

\usepackage{conecta}
\usepackage{geometry}
\usepackage{makeidx}
\usepackage{tikz}
%\usepackage{kpfonts}
\usepackage[explicit]{titlesec}
\usepackage{titletoc,tocloft}

% ---
% Pacotes fundamentais 
% ---
\usepackage{lmodern}			% Usa a fonte Latin Modern
\usepackage[T1]{fontenc}		% Selecao de codigos de fonte.
\usepackage[utf8]{inputenc}		% Codificacao do documento (conversão automática dos acentos)
\usepackage{indentfirst}		% Indenta o primeiro parágrafo de cada seção.
\usepackage{color}				% Controle das cores
\usepackage{graphicx}			% Inclusão de gráficos
\usepackage{microtype} 			% para melhorias de justificação
\usepackage{setspace}
% ---

% ---
% Pacotes adicionais, usados apenas no âmbito do Modelo Canônico do abnteX2
% ---
\usepackage{lipsum}				% para geração de dummy text
% ---

% ---
% Pacotes de citações
% ---
\usepackage[english,hyperpageref]{backref}	 % Paginas com as citações na bibl
\usepackage[alf]{abntex2cite}	% Citações padrão ABNT

% --- 
% CONFIGURAÇÕES DE PACOTES
% --- 


% ---
% Informações de dados para CAPA e FOLHA DE ROSTO
% ---
\titulo{Conecta Cientifica}
\autor{Grupo Oscar - Engenharia de Software Noturno}
\local{Brazil}
\data{2023}
\instituicao{Alunos Unifesp}
%\tipotrabalho{Software Manual}
\tipotrabalho{Change Log}
% O preambulo deve conter o tipo do trabalho, o objetivo, 
% o nome da instituição e a área de concentração 
\preambulo{\imprimirtitulo -- \imprimirtipotrabalho}
% ---

% ---
% Configurações de aparência do PDF final

% alterando o aspecto da cor azul
\definecolor{verdeClaro}{RGB}{60, 144, 110}
\definecolor{verdeEscuro}{RGB}{43,103,79}
\definecolor{cinza}{RGB}{50,62,72}

% informações do PDF
\makeatletter
\hypersetup{
	%pagebackref=true,
	pdftitle={\@title}, 
	pdfauthor={\@author},
	pdfsubject={\imprimirpreambulo},
	pdfcreator={PDF},
	pdfkeywords={paint}{manual}, 
	colorlinks=true,       		% false: boxed links; true: colored links
	linkcolor=cinza,          	% color of internal links
	citecolor=cinza,        		% color of links to bibliography
	filecolor=magenta,      		% color of file links
	urlcolor=cinza,
	bookmarksdepth=4
}
\makeatother
% --- 

% --- 
% Espaçamentos entre linhas e parágrafos 
% --- 

% O tamanho do parágrafo é dado por:
\setlength{\parindent}{0cm} %{1.25cm}

% Controle do espaçamento entre um parágrafo e outro:
\setlength{\parskip}{0cm}  % tente também \onelineskip


% ----
% Início do documento
% ----
\begin{document}
	
	% Seleciona o idioma do documento (conforme pacotes do babel)
	\selectlanguage{english}
	
	% Retira espaço extra obsoleto entre as frases.
	\frenchspacing 
	
	% ----------------------------------------------------------
	% ELEMENTOS PRÉ-TEXTUAIS
	% ----------------------------------------------------------
	\pretextual
	
	% ----------------------------------------------------------
	% ELEMENTOS TEXTUAIS
	% ----------------------------------------------------------
	\textual
	\pagestyle{meuestilo}
	
	\chapter*{Changelog SP1 - Conecta Cientifica v0.3.0} 
	
	\subsection*{\textbf{Features:}}
	\begin{itemize} \setlength\itemsep{0em}
		
		\item Implementação da página inicial.
		\item Implementação da página de login.
		\item Implementação da página de registro.

	\end{itemize}

	\subsection*{\textbf{Bugfix:}}
	\begin{itemize} \setlength\itemsep{0em}

		\item   

	\end{itemize}


	

	\chapter*{Changelog SP2 - Conecta Cientifica v0.12.4} 
	
	\subsection*{\textbf{Features:}} % novas features feitas
	\begin{itemize} \setlength\itemsep{0em}
		
		\item Criação do changelog.
		\item Implementação do cadastro e login usando o Google.
		\item Refatoração de View e Forms.
		\item Refatoração das rotas do site
		\item Refatoração da estilização da página de login
		\item Adicionar condições para mostrar cada link do header dependendo da URL
		\item Mudar cores dos botões
		\item Refatoração: Modificado DB para Postgres em nuvem
		\item Estruturação e estilização da página de perfil

	\end{itemize}

	\subsection*{\textbf{Bugfix:}} % correcoes feitas
	\begin{itemize} \setlength\itemsep{0em}

		\item Correção de bugs do login com o Google: carregamento da página (a página estava carregando apenas em abas anônimas, agora carrega em qualquer modo)
		\item Correção posicionamento do footer - acompanha a resolução da tela, sem sobrepor o conteúdo do body
		\item Mover scripts CSS dos botões de login/cadastro com o Google para o arquivo style.css
		\item Corrigido bug de rotas relacionado a API de login do google

	\end{itemize}


	\chapter*{Changelog SP3 - Conecta Cientifica v0.21.4} 
	
	\subsection*{\textbf{Features:}}
	\begin{itemize} \setlength\itemsep{0em}
		
		\item Elaboração design página exibição de projetos.
		\item Estruturar e estilizar página de exibição de projeto.
		\item Estruturar e estilizar página de projeto.
		\item Edição de informações do perfil.
		\item Criar um adapter para a conexão com o Lattes.
		\item Criar um mock que simule a API do Lattes.
		\item Logoff do usuário.
		\item Criar pipeline e colocar os testes nela.
		\item Criação do modelo de projeto.

	\end{itemize}

	\subsection*{\textbf{Bugfix:}}
	\begin{itemize} \setlength\itemsep{0em}

		\item   

	\end{itemize}

	\chapter*{Changelog SP4 - Conecta Cientifica v1.0.0} 
	
	\subsection*{\textbf{Features:}}
	\begin{itemize} \setlength\itemsep{0em}
		
		\item Refatorar estilização perfil usuário.
		\item Página de criação de projeto.
		\item Estilização página de criação de projetos.
		\item Corrigir testes quebrados.
		\item Rodar pipeline a cada commit ao invés de apenas no PR pra master.
		\item Ajustes página de perfil do usuário.
		\item Condicionar exibição das páginas a depender se a pessoa está logada ou não.
		\item Implementar testes unitários das novas funcionalidades.
		\item Finalizar o adapter para a conexão com o lattes.
		\item Exibir mensagens de alerta na página.
		\item Condicionar aplicação pra quando o usuário está ou não logado, criar variável para essa validação.

	\end{itemize}

	\subsection*{\textbf{Bugfix:}}
	\begin{itemize} \setlength\itemsep{0em}

		\item Edição do perfil de usuário.

	\end{itemize}

	\chapter*{Changelog SP5 - Conecta Cientifica v1.X.X} 
	
	\subsection*{\textbf{Features:}}
	\begin{itemize} \setlength\itemsep{0em}
		
		\item Criação de caixas de mensagens para login e registro.
		\item Adição do user que criou o projeto aos dados do projeto.
		\item Criação da página de edição de projeto.
		\item Criação da página de exclusão de projeto.
		\item Conexão da página de exibição de projetos (feed) com os criados no banco.
		\item Adição da opção de inscrição ou desinscrição de um projeto.
		\item Criada a página para aprovação das solicitações de inscrições em projetos.
		\item Adição de filtros no feed de projetos.
		\item Condicionar exibição de links com autenticação do usuário.
		\item Estilização das páginas de criação, exclusão e edição de projetos.
		\item Melhorar exibição das tags do projeto na página de projeto.

	\end{itemize}

	\subsection*{\textbf{Bugfix:}}
	\begin{itemize} \setlength\itemsep{0em}

		\item Corrigir campos dos forms do app accounts, para receberem estilização correta.
		\item Refazer estilização da página de projeto, que havia se perdido.

	\end{itemize}



	% ---
	% Finaliza a parte no bookmark do PDF
	% para que se inicie o bookmark na raiz
	% e adiciona espaço de parte no Sumário
	% ---
	
	\phantompart
	
	% ----------------------------------------------------------
	% ELEMENTOS PÓS-TEXTUAIS
	% ----------------------------------------------------------
	\postextual
	
	\phantompart
	
\end{document}
